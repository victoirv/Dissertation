\chapter[Conclusions and Future Work]{Conclusions and Future Work}

\section{Conclusions}
This work examined the behavior and statistical properties of ion mass density in the Earth's plasmatrough. By using both linear and nonlinear models, different forms of behavior could be analyzed. It began by verifying the results of previous studies, and extended them in scope to determine if the results held true for more general cases, or cases where the data was not ideal or complete. 

It was shown in Section \ref{sec:epochs} that while equatorial mass density (\req) did vary significantly on an hourly timescale when a drop in \dst\ was observed, it was less significant on a daily timescale. It was also shown that increases in \req\ were not, on average, preceded or succeeded by any significant change in the examined solar wind or geomagnetic variables, despite prior results that looked at a few selected events.  This suggests that typical \req\ enhancements are the result of plasmasphere expansion rather than being driven by geomagnetic activity.

The results of \cite{Takahashi2010SolarCycleVariation} were verified in Section \ref{sec:f107dep} showing that \req\ and \f\ have a strong correlation, which is stronger over longer timescales such as 27 days than it is over an hourly timescale. It was then shown that this connection seems to affect the behavior of \req\ most during periods of strong solar activity, leading to large \req\ reactions to \dst\ drops for high values of \f.  It was also shown that \req\ behaved significantly different before and after events based on the value of \f\ at the onset of a \req\ event or a \dst\ event.  This suggests that large values of \f\ act as a nonlinear precondition for \req\ to significantly react to geomagnetic activity.

A similar analysis for $B_z$ in Section \ref{sec:bzdep} showed much less significance, but that a southward $B_z$ immediately following event onset leads to impeded \req\ growth.

Section \ref{sec:ClassifyResults} shows that it is very much possible to distinguish an event onset from the three hours preceding it based on a non-linear classification model using solar wind conditions, suggesting that events do not occur independent of their surroundings. It was also shown that distinguishing the day of an event from the three days preceding it is possible, though not as accurate.

Section \ref{sec:ForecastResults} shows that the day of an event can be predicted based on the four days of solar and geomagnetic conditions leading up to it using a binary classification model. While it does well at predicting events, it also predicts a number of false positives which are shown to be distributed similarly across the input variables as the true positives.  This suggests that some other information is needed to distinguish conditions leading up to events.



\section{Future Work}

This work contained many exploratory elements, the results of some for which, though negative, are also worth briefly mentioning. For linear models of \req\ based on solar wind and geomagnetic variables, a high degree of collinearity is present preventing the addition of more variables from improving the model. Whether the signal remaining after subtracting the correlated pieces is just noise or could be accounted for via a variable this study did not use is a topic for future research.

The statistical models used were somewhat simplistic compared to many of the common MHD and nonlinear models used in current studies of the magnetosphere and plasmasphere.  Though that level of generalization was largely intentional, extra degrees of specificity and accounting for non-linear effects could potentially improve the models. Similarly the results in Section \ref{sec:ForecastResults}, which found that incorrect predictions came from roughly the same values as correct predictions, suggest more information could be added to the model to improve prediction quality.

The detection rate of the Alfvén waves used to determine the plasmatrough mass density could potentially be improved via better satellite coverage (e.g. by the upcoming GOES-R satellites), better equipment onboard the satellites, or some other technique that would then aid the lacking data availability that this study had to overcome. 

This study could also be extended in time. It focused on using only GOES 6 in order to eliminate any bias from combining data from different satellites and dealing with overlap, different equipment, etc. A study that dealt with overcoming those issues could extend the analysis to at least double the time length using the dataset provided by \cite{Denton}, and possibly even further using the newer GOES satellites by reproducing the techniques outlined by \cite{Takahashi2006MassDensityInferred} and \cite{Kondrashov2014ReconstructionOfGaps}. 
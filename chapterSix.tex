\chapter[Conclusions and Future Work]{Conclusions and Future Work}

\section{Conclusions}
This work examined the behavior and statistical properties of ion mass density in the Earth's plasmatrough. By using both linear and nonlinear models, different forms of behavior could be analyzed. It began by verifying the results of previous studies, and extended them in scope to determine if the results held true for more general cases, or cases where the data was not ideal or complete. 

It was shown that while equatorial mass density (\req) did not vary significantly on an hourly timescale when a drop in \dst\ was observed, it did vary significantly between the day of an event onset and the day immediately following an onset.

It was shown that increases in \req\ were not, on average, preceded or succeeded by any significant change in the examined solar wind or geomagnetic variables, despite prior results that looked at a few selected events. 

It was verified that \req\ and \f\ have a strong correlation, which is stronger over longer timescales such as 27 days than it is over an hourly timescale. It was then shown that this connection seems to affect the behavior of \req\ most during periods of strong solar activity, leading to large \req\ reactions to \dst\ drops for high values of \f.  It was also shown that \req\ behaved significantly different before and after events based on the value of \f\ at the onset of a \req\ event or a \dst\ event.  



\section{Future Work}

This work contained many exploratory elements, the results of which, though negative, are also worth briefly mentioning. For linear models of \req\ based on solar wind and geomagnetic variables, a high degree of collinearity is present preventing the addition of more variables from improving the model. Whether the signal left after subtracting the correlated pieces is just noise or could be accounted for via a variable this study did not use is a topic for future research.

The statistical models used were somewhat simplistic compared to many of the common MHD and nonlinear models used in current studies of the magnetosphere and plasmasphere.  Though that level of generalization was largely intentional, extra degrees of specificity and accounting for non-linear effects could potentially improve the models.

The detection rate of the Alfvén waves used to determine the plasmatrough mass density could potentially be improved via better satellite coverage (e.g. by the upcoming GOES-R satellites), better equipment, or some other technique that would then aid the lacking data availability that this study had to overcome. 
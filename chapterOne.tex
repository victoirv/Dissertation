
%% This file represents a sample first chapter of the main body of the dissertation
%%
%%**********************************************************************
%% Legal Notice:
%% This code is offered as-is without any warranty either
%% expressed or implied; without even the implied warranty of
%% MERCHANTABILITY or FITNESS FOR A PARTICULAR PURPOSE!
%% User assumes all risk.
%% In no event shall any contributor to this code be liable for any damages
%% or losses, including, but not limited to, incidental, consequential, or
%% any other damages, resulting from the use or misuse of any information
%% contained here.
%%**********************************************************************
%%
%% $Id: chapterOne.tex,v 1.6 2006/08/24 21:13:45 Owner Exp $
%%

% A first, optional argument in [ ] is the title as displayed in the table of contents
% The second argument is the title as displayed here.  Use \\ as appropriate in
%   this title to get desired line breaks
\chapter[Introduction]{Introduction}

\section{Background}

\subsection{Magnetosphere}

\subsubsection{Discovery}

\subsubsection{Processes}

Storms, substorms, etc.

The dynamic processes of Earth's magnetosphere and their various impacts on the planet and its inhabitants have been studied for centuries: from Celsius and Hiorter who noted a correlation between compass orientation and aurora \cite{Maunder} , to studies of The Carrington Event \cite{Carrington}, to Chapman and Ferraro's "A new theory of magnetic storms" \cite{Chapman}, scientists have been measuring the currents induced by the magnetosphere and trying to explain the effects they saw. With the advent of space flight, in-situ measurements of the magnetosphere became possible and long-term data collection became prevalent, allowing forecasting to become feasible. Our current computational technology, combined with over 50 years' worth of satellite and ground based measurements \cite{HistMagnetometer}, allows for a much stronger statistics-based forecasting method to be performed and long-term analyses of the capabilities of computationally intensive forecasting methods.

Geomagnetic storms occur when the solar wind interacts with the Earth's magnetosphere in such a way as to produce significant disruptions in its normal, quiet-time, behavior. It is generally defined by a significant change to the magnetic field measured by multiple ground-based magnetometer measurements from stations spread around the world, in the case of the $K_P$ index, or around the geomagnetic equator in the case of the Disturbance storm-time ($D_{st}$) index. By using these indices, storms can be classified into categories of severity \cite{NOAAScale}. The definition of storms in the literature varies slightly between authors \cite{Yermolaev}, but most agree that sustained and abnormally perturbed near-earth magnetic field strengths over several hours or more constitutes a geomagnetic storm \cite{StormDefinition}. 

Geomagnetic storms can have significant impacts on Earth and space systems, from inducing currents in large power grids to harming satellite circuitry and onboard data \cite{1989Storm}. Because of the potential damage of such events, any ability to forecast a storm could allow operators to prevent or mitigate problems in their systems. Becuase of the large correlation of CMEs with geomagnetic storms \cite{Yermolaev}, it can be estimated that our forewarning time is the difference between observing a CME (via visual or X-ray methods) and its propagation time plus magnetospheric interaction time. This time can be anywhere from one to five days, depending on the speed of the CME and how it interacts with the interplanetary medium \cite{StormSources}. With a light delay of only eight minutes, this is ample time to see a storm approaching Earth and for operators to react, but a problem lies in the fact that storms are poorly predicted with such lead times \cite{WeigelDecision}. Some storms have slow onsets, some spike suddenly; some have high velocities, and some coincide with large amounts of high-energy particles; no single factor has yet proven to be a good predictor for storms, and while prediction has gradually improved over the years, there remains room for further study. 

Initial forecasts were based on an observed time delay between sunspots and geomagnetic storms \cite{SunspotStorms}. It then advanced to a basic theory involving electromagnetic interactions in the magnetosphere \cite{Chapman}. There now exist entire services dedicated to executing MHD-based models of the magnetosphere \cite{CCMC}, as well as multi-year, multi-institution efforts to survey the general statistics of modeling and forecasting of extreme events \cite{ExtremeEvents}.

The convergence of the advancement in both statistical and MHD-based simulation has led to a situation where the scientific community has the capacity for monitoring space weather in real time, and forecasting the near-Earth effects.  There have been efforts to test the forecast performance of select models over a small number of geomagnetic events \cite{ANNforecast,StormModel,StatCompStorms,Yermolaev}. However no research has been done that involves the analysis of long-term forecasting performance of these models and comparison of the results with existing methods.

In order to effectively propose a research topic, the current limits of knowledge in the field must be mapped out. This includes current numerical methods being used for forecasts, current thoughts on the best types of models for the relevant systems, and what simulations have recently been conducted as well as any details they may have left out.

Leave off with statement on how outer magnetosphere influences plasmasphere.

\subsection{Plasmasphere}

\subsubsection{Discovery}

\subsubsection{Processes}

Plumes, erosion, etc.

Leave off with list of things that are not well understood and how work in thesis approaches them.

\subsection{Statistical Modeling of Magnetopshere and Plasmasphere}

Make connection to each paper with a section in chapter 2.


